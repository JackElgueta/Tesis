\chapter[Aplicando Minería de Datos con RapidMiner]{Aplicando Minería de Datos con RapidMiner}
\label{ch:desmin}
\subsection{Identificación de datos y variables}

Los datos a utilizar en los modelos, son variables extraídas del análisis de los constructos de la sección anterior, además se incluyen variables analizadas al principio del documento sobre la deserción en Chile

\begin{table}[H]
	\begin{center}
		\begin{tabular}{|l|l|l|}
			\hline
			Variable & Tipo & Descripción \\
			\hline \hline
			&  & \\ \hline
			Distancia hogar-institución &  & \\ \hline
			Genero &  & \\ \hline
			Edad &  & \\ \hline
			Tipo establecimiento de origen &  & \\ \hline
			Notas enseñanza media & & \\ \hline
			Promedio PSU lenguaje y matemáticas &  & \\ \hline
			Prioridad de postulación a carrera & & \\ \hline
			Reingresante & & \\ \hline
			Trabajador & & \\ \hline
			\hline \hline
			
			Tipo Institución &  & \\ \hline
			Acreditación Institucional &  & \\ \hline
			Jornada &  & \\ \hline
			Arancel anual & & \\ \hline
			Empleabilidad al 1er año & & \\ \hline
			Ingreso promedio 4to año & & \\ \hline
			Carrera acreditada & & \\ \hline
			\hline \hline
			
			Semestres cursados & & \\ \hline
			Taller profesional realizados & & \\ \hline
			Ha realizado intercambio & & \\ \hline
			Taller certificación realizados & & \\ \hline
			Taller deportivo o cultural & & \\ \hline
			Tiene beca deportiva & & \\ \hline
			Tiene beca de estudio &  & \\ \hline
			Tiene beca de alimento & & \\ \hline
			Tiene crédito &  & \\ \hline
			Promedio notas acumuladas &  & \\ \hline
			Ramos aprobados por semestre & & \\ \hline
			Causales acumuladas & & \\ \hline
			Ranking del alumno & & \\ \hline
			Prácticas realizadas & & \\ \hline
			Ha realizado voluntariado social & & \\ \hline
			
		\end{tabular}
		\caption{Tabla muy sencilla.}
		\label{tabla:sencilla}
	\end{center}
\end{table}


\subsection{Modelamiento}
