\chapter[Aplicando Minería de Datos con RapidMiner]{Aplicando Minería de Datos con RapidMiner}
\label{ch:desmin}
\subsection{Identificación de datos y variables}

Los datos a utilizar en los modelos, son variables extraídas del análisis de los constructos de la sección anterior, además se incluyen variables analizadas al principio del documento sobre la deserción en Chile

\begin{table}[H]
	\begin{center}
		\begin{tabular}{|l|l|l|}
			\hline
			Variable & Tipo & Descripción \\
			\hline \hline
			&  & \\ \hline
			Comuna Residencia &  & \\ \hline
			Genero &  & \\ \hline
			Edad &  & \\ \hline
			Tipo establecimiento de origen &  & \\ \hline
			Notas enseñanza media & & \\ \hline
			PSU Matemáticas &  & \\ \hline
			PSU Lenguaje &  & \\ \hline
			PSU Historia &  & \\ \hline
			PSU Ciencias &  & \\ \hline
			Prioridad de postulación a carrera & & \\ \hline
			Reingresante & & \\ \hline
			Trabajador & & \\ \hline
			\hline \hline
			
			Tipo Institución &  & \\ \hline
			Acreditación Institucional &  & \\ \hline
			Jornada &  & \\ \hline
			Arancel anual & & \\ \hline
			Empleabilidad al 1er año & & \\ \hline
			Ingreso promedio 4to año & & \\ \hline
			Institución con intercambio & & \\ \hline
			Carrera acreditada & & \\ \hline
			Talleres deportivos & & \\ \hline
			Talleres culturales & & \\ \hline
			Becas deportivas & & \\ \hline
			Convenios culturales & & \\ \hline
			Certificación & & \\ \hline
			\hline \hline
			
			Beca &  & \\ \hline
			Beca alimento & & \\ \hline
			Crédito &  & \\ \hline
			Promedio notas acumuladas &  & \\ \hline
			Ramos reprobados por semestre & & \\ \hline
			Causales acumuladas & & \\ \hline
			Ranking carrera & & \\ \hline
			
		\end{tabular}
		\caption{Tabla muy sencilla.}
		\label{tabla:sencilla}
	\end{center}
\end{table}


\subsection{Modelamiento}
