\chapter[Conclusiones y recomendaciones]{Conclusiones y recomendaciones}
\label{ch:conclu}

A partir del estudio de la deserción según estadísticas del SIES, el tema de deserción universitaria, no es un fenómeno aislado, aproximadamente el 50 \%  de los alumnos que ingresan a un programa de estudios universitario, no termina la carrera, preocupando a varios actores del sistema.\\

Los datos recopilados por el SIES, orientan en cierto sentido el panorama que se a presentado entre los años 2010 y 2014 sobre el tema deserción, tomando variables globales como institución, género, jornada, establecimiento de origen y acreditación.\\

El concepto de inteligencia del cliente, entrego herramientas para analizar las necesidades y relacionamiento que tiene el alumno con la universidad, esto ayudo a identificar a través de entrevistas, variables influyentes en continuar y/o desertar de la carrera. Las variables determinadas con inteligencia del cliente junto a las variables de información del SIES, ayudaron al proceso de inteligencia de negocio, las cuales fueron extraídas, transformadas y cargadas en una base analítica, para finalizar aplicando el concepto minería de datos y generar los modelos con la herramienta RapidMiner.\\