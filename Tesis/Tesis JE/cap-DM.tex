\chapter[Minería de Datos]{Minería de Datos}
\label{ch:dm}

\section{Minería de Datos}

El proceso de minería de datos o exploración de datos es un proceso, considerado una etapa de un proceso mayor llamado ``Descubrimiento de Conocimiento en Base de Datos'' (KDD, del inglés, Knowledge Discovery in Databases), ``proceso no trivial de identificar patrones válidos, novedosos, potencialmente útiles y comprensible a partir de los datos''. \footnote{Fayyad et al. 1996}\\

Usualmente ambos conceptos son intercambiables, pero se entiende por KDD al proceso de encontrar información y/o patrones útiles en los datos. En cambio minería de datos, es el uso de algoritmos para extraer información y/o patrones derivados dentro del proceso KDD.\\


Dentro de la definición del proceso KDD se pueden desprender los siguientes conceptos:

\begin{itemize}
    \item Patrones válidos:
     Los patrones deben ser precisos y verdaderos para nuevos datos, similar al principio de inducción matemática.
    \item Novedosos:
    Los datos deben aportar algo nuevo al usuario, un conocimiento. Esto es la esencia de la minería de datos.
    \item Útil:
    Los datos deben conducir a acciones por parte de los usuarios receptores.
    \item Comprensible:
    Los datos deben ser posible de interpretar y por ende comunicar.
\end{itemize}

En la Figura \ref{fig:kdd}, se puede apreciar el proceso de KDD.\\

\begin{figure}[H]
\begin{minipage}{\textwidth} 
\centering 
\includegraphics[width=10cm,height=5cm] {kdd.png} 
\caption[Proceso KDD]{Proceso KDD~\footnote{Imagen extraída de http://www.scielo.org.co articulo ``Aplicación del proceso de KDD en el contexto de bibliomining: El caso Elogim''}}
\label{fig:kdd}
\end{minipage}
\end{figure}

\subsection{Fases del Descubrimiento del Conocimiento}

El proceso KDD puede ser representado en 7 objetivos, los cuales se logran en diferentes fases, estos objetivos son:

\begin{enumerate}
    \item Determinar las fuentes de información que pueden ser útiles y dónde conseguirlas.
    \item Diseñar un esquema de almacenamiento de datos, un Data Warehouse.
    \item Implementación del almacén de datos que permita la navegación y visualización previa de los datos para el análisis.
    \item Selección, limpieza y transformación de los datos a analizar.
    \item Seleccionar y aplicar el método de minería de datos apropiado.
    \item Evaluación, interpretación, transformación y representación de los patrones extraídos.
    \item Difusión y uso del nuevo conocimiento.
    
\end{enumerate}

\subsubsection{Recogida de Datos}

La primeras fases del KDD determinan que las fases sucesivas sean capaces de extraer conocimiento válido y útil a partir de la información original.\\

En la fase de recogida de datos se sigue los siguientes pasos:

\begin{itemize}
    \item Selección de datos: 
    Por lo general la información que se quiere investigar sobre un cierto dominio de la organización se encuentra en bases de datos y fuentes internas como externas.
    \item Pre-proceso o limpieza de datos:
    En este paso se debe eliminar el mayor número de datos erróneos o inconsistentes e irrelevantes, esto también se conoce como limpieza y selección.
    \item Transformación de datos:
    Una vez limpiados y seleccionados los datos, se realiza un proceso de transformación que permite homologar la información, dando como resultado un conjunto de filas y columnas denominado ``Vista Minable'', con el fin de dejar los datos preparados para la modelización.
\end{itemize}

\subsubsection{Modelamiento}

El modelamiento o minería de datos es la etapa principal del KDD, aquí se determinan patrones y modelos para descubrir el conocimiento. Existen dos tipos de modelos y diferentes tareas para cada modelo.

\begin{enumerate}
    \item Tareas del Modelo Predictivo\\
    Son aquellas que buscan patrones que ayuden a predecir el comportamiento o tendencia de uno o varios valores.\\
    
    \begin{itemize}
        \item Clasificación:
        Los datos o registros son agrupados en clases, los cuales pueden tomar valores discretos. Su objetivo es determinar patrones en los registros, para identificar o predecir a que clase pertenecen los registros nuevos.
        
        \item Regresión:
        Se usa una regresión para predecir los valores ausentes de una variable basándose en su relación con otras variables del conjunto de datos.
    \end{itemize}
    
    \item Tareas del Modelo Descriptivo\\
    Son aquellas que exploran las propiedades de los datos analizados para encontrar patrones entre ellos.\\
    
    \begin{itemize}
        \item Agrupamiento (Clasificación no supervisada):
        Es similar a la clasificación, excepto que los grupos no son predefinidos. El objetivo es segmentar un conjunto de datos en grupos que pueden ser disjuntos o no. Los grupos se forman basados en la similaridad de los datos en ciertas variables. Como los grupos no son dados a priori se debe dar una interpretación de los grupos que se forman.
        
        \item Correlaciones:
        Identifica el grado de similitud de dos variables numéricas.
        \item Asociación:
        Las asociaciones entre dos atributos ocurre cuando la frecuencia de que se den dos valores determinados de cada uno conjuntamente es relativamente alta.
        \item Reglas de asociación secuenciales:
        Es una variante de asociación que utiliza la variable tiempo para identificar correlación.

        
    \end{itemize}
    
\end{enumerate}


Para estos modelos y tareas existen diferentes técnicas de minería de datos las cuales se pueden aplicar, a continuación se muestra un cuadro de las técnicas y tareas que utilizan, Figura \ref{fig:tecnicas}


\begin{figure}[H]
\begin{minipage}{\textwidth} 
\centering 
\includegraphics[width=12cm,height=7cm] {tecnicas.png}
\caption[Técnicas de minería de datos]{Técnicas de minería de datos~\footnote{Imagen extraída del libro ``Introducción a la Minería de Datos'', pag. 148, cap. 6}}
\label{fig:tecnicas}
\end{minipage}
\end{figure}

\subsubsection{Evaluación}

La fase de evaluación tiene como objetivo probar y validar el modelo creado en las fases anteriores. Para realizar las pruebas del modelo, se divide el set de datos en dos grupos, un grupo de entrenamiento de los datos, que ayuda a identificar la predicción que se espera, y el otro grupo es de prueba se valida la predicción y se analiza el porcentaje de acierto de dicha predicción. Por lo general se obtienen varios modelos aplicando las diferentes técnicas de minería de datos, los cuales son comparados buscando aquel que se ajuste mejor al problema. Si ninguno de los modelos alcanza los resultados esperados, debe
alterarse alguno de los pasos anteriores para generar nuevos modelos.

\section{Técnicas de Minería de Datos}

En el capitulo anterior se explico que existen diferentes técnicas y tareas en minería de datos para los distintos tipos de modelamiento.\\

El modelamiento acorde a este trabajo, corresponde a un modelo predictivo, en el cual se utilizarán cuatro técnicas en RapidMiner, para generar cuatro modelos, estas técnicas son:\\

\begin{itemize}
	\item Regresión Lineal:\\
Es una técnica utilizada para la predicción numérica. Es una medida estadística que determina la relación ente una variable dependiente y una serie de variables independientes.\\
La regresión lineal intenta modelar la relación entre una variable escalar y una o más variables explicativas ajustando una ecuación lineal a los datos observados. Por ejemplo, uno podría querer relacionar los pesos de los individuos con sus alturas usando un modelo de regresión lineal \cite{rl}.

	\item Árboles de decisión:\\


	\item Redes Neuronales:\\


	\item Máquinas de Vectores de Soporte:\\


\end{itemize}  

\section{RapidMiner}

RapidMiner es una herramienta de código abierto, usado para la minería de datos. Utiliza un ambiente gráfico que permite combinar distintos operadores y generar procesos de tratamiento y/o análisis de datos\\

\section{Conclusiones Marco Teórico}

A partir del estudio de la deserción según estadísticas del SIES, el tema de deserción universitaria, no es un fenómeno aislado, aproximadamente el 50 \%  de los alumnos que ingresan a un programa de estudios universitario, no termina la carrera, preocupando a varios actores del sistema.\\

Los datos recopilados por el SIES, orientan en cierto sentido el panorama que se a presentado entre los años 2010 y 2014 sobre el tema deserción, tomando variables globales como institución, género, jornada, establecimiento de origen y acreditación.\\

El concepto de inteligencia del cliente, entrega herramientas para analizar las necesidades y relacionamiento que tiene el alumno con la universidad, esto ayudará para identificar a través de entrevistas, variables influyentes en continuar y/o desertar de la carrera. Las variables determinadas ayudarán al proceso de inteligencia de negocio, las cuales serán extraídas, transformadas y cargadas en una base analítica, para finalizar aplicando el concepto minería de datos y generar los modelos con la herramienta Rapid Miner.












